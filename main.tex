\documentclass[
  normalmargins,
  10pt,
  openany,
  onehalfspacing,
  UTF8
]{my-format}
% packages
\usepackage{ctex}
\usepackage[colorlinks]{hyperref} % for links
\usepackage[dvipsnames]{xcolor}
\usepackage{acro}
\usepackage{fancyhdr} 
\pagestyle{fancy}

\DeclareAcronym{CC}{
  short=CC,
  long=Cross-Correlation,
    }
\DeclareAcronym{CP-OTDR}{
  short=CP-OTDR,
  long=Chirped Pulse OTDR,
    }
\DeclareAcronym{DFB}{
  short=DFB,
  long=Distributed Feedback,
    }
\DeclareAcronym{EDFA}{
  short=EDFA,
  long=Erbium Doped Fiber Amplifier,
    }
\DeclareAcronym{FBG}{
  short=FBG ,
  long=Fiber Bragg Grating,
    }
\DeclareAcronym{FFT}{
  short=FFT,
  long=Fast Fourier Transform,
    }
\DeclareAcronym{FP}{
  short=FP,
  long=Fabry-Pérot,
    }
\DeclareAcronym{FUT}{
  short=FUT,
  long=Fiber Under Test,
    }
\DeclareAcronym{FWHM}{
  short=FWHM,
  long=Full Width at Half Maximum,
    }
\DeclareAcronym{MWE}{
  short=MWE,
  long=Maxwell's Equations,
    }
\DeclareAcronym{NIR}{
  short=NIR,
  long=Near-Infrared,
    }
\DeclareAcronym{NLSE}{
  short=NLSE,
  long=Nonlinear Schrodinger equation,
    }
\DeclareAcronym{OFDR}{
  short=OFDR,
  long=Optical Frequency Domain Reflectometry,
    }
\DeclareAcronym{OPD}{
  short=OPD,
  long=Optical Path Difference,
    }
\DeclareAcronym{OTDR}{
  short=OTDR,
  long=Optical Time Domain Reflectometry,
    }
\DeclareAcronym{P-OTDR}{
  short=P-OTDR,
  long=Polarization OTDR,
    }
\DeclareAcronym{PIC}{
  short=PIC,
  long=Photonic Integrated Circuit,
    }
\DeclareAcronym{RFGA}{
  short=RFGA,
  long=Random Fiber Grating Array,
    }
\DeclareAcronym{SOA}{
  short=SOA,
  long=Solid State Optical Amplifier,
    }
\DeclareAcronym{SOP}{
  short=SOP,
  long=State of Polarization,
    }
\DeclareAcronym{SOTA}{
  short=SOTA,
  long=State of the Art,
    }
\DeclareAcronym{SSFM}{
  short=SSFM,
  long=Split-Step Fourier Method,
    }
\DeclareAcronym{TDR}{
  short=TDR,
  long=Time Doman Reflectometry,
    }
\DeclareAcronym{UV}{
  short=UV,
  long=Ultraviolet,
    }
\DeclareAcronym{WSS}{
  short=WSS,
  long=Wavelength Selective Switch,
    }
    
    
    
    
\usepackage[many]{tcolorbox}    	% for COLORED BOXES (tikz and xcolor included)
\newtcolorbox{boxA}{
    fontupper = \bf,
    boxrule = 1.5pt,
    colframe = black % frame color
}
\newcommand{\A}{\textcolor{red}{A}} 
\newcommand{\E}{\textcolor{red}{E}} 
\newcommand{\greencheck}{\textcolor{green}{\checkmark}}
\newcommand{\redcross}{\textcolor{red}{x}}


\newcommand{\betag}{\textcolor{ForestGreen}{\beta}} 
\newcommand{\CITE}{\textcolor{magenta}{CITE?!?!?!?!?!}} 
\newcommand{\real}{\mathfrak{Re}}
\newcommand{\FT}{\mathfrak{F}}
\newcommand{\IFT}{\mathfrak{F}^{-1}}



\usepackage{amsmath}
\usepackage{amsfonts}
\usepackage{graphicx} % for embedding graphics
\usepackage{booktabs} % for pretty tables
\usepackage{pgfplots}
\pgfplotsset{compat=1.18,width=11cm}
\usetikzlibrary{math}
\usepackage[all]{nowidow}
\usepackage{pdflscape}
\usepackage[final]{pdfpages}
\usepackage{afterpage}
\usepackage[nottoc]{tocbibind}
\usepackage{amssymb}
\usepackage{changepage}

\newcommand{\degs}{^{\circ}}
\newcommand{\inv}{$^{-1}$}


% author data
\author{Ole Krarup}
\title{\Large 非线性薛定谔方程简明教程}
\date{}

\begin{document}
 
  \frontmatter    
        
    \maketitle
    \begin{center}
        \text{\Large 很惭愧,只做了一点微小的贡献}
    \end{center}
    \begin{flushright}
      \text{\Large 翻译:蓝睿博}
    \end{flushright}
    \begin{flushright}
      \text{\Large 2024年,南京}
    \end{flushright}
    \addtocounter{page}{1}
    
   
   
    \addcontentsline{toc}{chapter}{Table of Contents}
    \tableofcontents
    
    \listoffigures
    \listoftables
    
    %
\color{white}
    \begin{tiny}
        \ac{CC}
        \ac{CP-OTDR}
        \ac{DFB}
        \ac{EDFA}
        \ac{FBG}
        \ac{FFT}
        \ac{FP}
        \ac{FUT}
        \ac{FWHM}
        \ac{MWE}
        %\ac{NIR}
        \ac{NLSE}
        \ac{OFDR}
        \ac{OPD}
        \ac{OTDR}
        \ac{P-OTDR}
        %\ac{PIC}
        \ac{RFGA}
        \ac{SOA}
        \ac{SOP}
        \ac{SOTA}
        \ac{SSFM}
        \ac{TDR}
        %\ac{UV}
        %\ac{WSS}
        
    \end{tiny}
\color{black}
    %\printacronyms
    %\addcontentsline{toc}{chapter}{List of Acronyms}
    
    \mainmatter
    
    \pagestyle{plain}
    \chapter{Introduction}
\label{ch:Introduction}
The Nonlinear Schr{\"o}dinger Equation (NLSE) in its generalized, scalar form describes how the normalized envelope $\A =\A(z,T)$ of the complex electric field $\E= \E(z,T) = \E_0\cdot \exp(-i(\betag(\omega_0)z-\omega_0T))$, oscillating with a carrier angular frequency, $\omega_0$, and carrier spatial frequency $\betag(\omega_0)$, evolves as it propagates through a medium, where attenuation, dispersion and a $\chi^{3}$ nonlinearity are present. Mathematically, it is given by

\begin{align}
    \label{eq:GNLSE}
    \partial_z \A = \frac{\alpha}{2}\A+i \sum_{n=2}^{\infty}i^n \frac{\betag_n}{n!}\partial_T^n\A  + i\gamma\left(1+\frac{i}{\omega_0}\partial_T  \right)\left( 
\A \int_{0}^{\infty} R(T_{delay})|\A(z,T-T_{delay})|^2 dT_{delay} \right),
\end{align}
where $\alpha$ is the power attenuation/gain coefficient, $\betag_n=\partial_\omega^n\betag(\omega)|_{\omega=\omega_0}$ are the coefficients of the Taylor expansion of the spatial frequency evaluated at $\omega = \omega_0$,  $\gamma$ is the nonlinearity parameter and $R(T_{delay})$ is the temporal response function of the nonlinearity at a time delay, $T_{delay}$, before the present time, $T$. Solving Eq.~\ref{eq:GNLSE} allows one to describe supercontinuum generation~\cite{supercontinuum_original_paper,NLSE_original}, solitons~\cite{soliton_first_theory,Soliton_experimental_first}, nonlinear noise in fiber telecommunications systems~\cite{poggiolini2014detailedanalyticalderivationgn} and other exotic optical phenomena with a plethora of scientific and industrial applications. 

\section{Goal}
This primer explores the constituent terms of Eq.~\ref{eq:GNLSE} and their interactions in a way that aims to develop an intuitive understanding of the mathematics and the underlying physics. To achieve this goal, discussions of more complicated effects, such as those involving the polarization of light, are omitted in favour of more detailed derivations and examples of purely scalar effects. Hopefully, this approach provides the reader, with the basic tools needed for analyzing common experimental results in nonlinear optics and tackling more advanced resources, papers and textbooks on the topic. 

\section{Availability}
This primer is freely available on \href{https://github.com/OleKrarup123/NLSE-primer}{GitHub} and continuously updated by the author in response to reader feedback. Readers are encouraged to submit questions, tips and suggestions to \href{yourfavouriteta@gmail.com}{YourFavouriteTA@gmail.com}. 


\section{Citation Policy}
This primer cites both original academic works on the explored topics, and links to items such as YouTube videos, personal web pages, online encyclopedia entries, interactive tools and similar material produced by hobbyists and professional researchers alike. The aim is to both provide the reader with a starting point for a comprehensive review of the formal literature necessary for independently writing a paper or thesis on nonlinear optics and a collection of high-quality, accessible explanations to deepen their understanding. 

\subsection{On citing this primer}
This primer contains no original research on the NLSE and should be viewed as a collection of detailed notes. It should not be cited as a source when writing a paper or thesis presenting new research on the NLSE. Instead, please cite the oldest original works on relevant topics, such as~\cite{soliton_first_theory} and~\cite{Soliton_experimental_first} in the case of solitons. However, if the aim is simply to familiarize readers with the NLSE to help them grasp original research on it, this primer can be cited in BibTeX format as~\cite{NLSE_primer}:\\
\begin{boxA}
@misc\{NLSE\_primer,  \\
author = "O. Krarup", \\
title = "\{A Primer on the Nonlinear Schrödinger Equation\}", \\
note = "Commit SHA: 9f1b93a", \\
url = \{https://github.com/OleKrarup123/NLSE-primer/blob/main/NLSE\_primer.pdf\}\}
\end{boxA}

Note that because this primer is freely available on GitHub and is continuously updated, the URL should link to the most recent version, while the "Commit SHA" should be that of the most recent commit the citation is made.     
    \chapter{数学和理论}
\label{ch:MathAndTheory}

本章介绍描述电磁波和理解方程~\ref{eq:GNLSE} 所需的基本数学工具。

\section{实数场和复数场}
%Explain that the real field is what determines how an electron moves and that the complex one is only used for mathematical convenience when dealing with phase shifts.
带电荷$q$、质量$m$的粒子在电场 $\Bar{\E}_r = \E_r \hat{x}$ 作用下的加速度为
\begin{align}
    \Bar{a} &=  \frac{q\E_r}{m}\hat{x}.
\end{align}
电场下标$r$表示这是“实”电场,它决定了带电粒子的加速度。而“复”电场则是一种数学工具,使得涉及电磁波的计算更为简便,并且可以始终从中恢复出实电场。例如,传播在体介质中的实电场波,其空间角频率 $\betag$ 依赖于时间角频率 $\omega$,可以表示为:\begin{align}
\label{eq:real_field}
    \E_r(z,t) &= |\E_0|\cos\left(\betag(\omega)z-\omega t+\phi\right) \\\nonumber 
     &=|\E_0| \real\left\{  \exp\left( i\betag(\omega)z-i\omega t+i\phi \right) \right\} \\ \nonumber 
     &= \real\left\{ \E_0 \exp\left( i\betag(\omega)z-i\omega t\right) \right\}  \\ \nonumber
     &= \real\left\{ \E(z,t) \right\}  \\ \nonumber
     &=  \frac{1}{2}\left( \E(z,t) + \E^*(z,t) \right).   
\end{align}
关于方程 \ref{eq:real_field} 的说明,请参见 \href{https://www.desmos.com/calculator/fgvozursrl}{此交互式图表}。
使用复电场在模拟相位变化和干涉效应时非常方便。例如,要在实际电场中引入相位移 $\phi_0$,必须通过以下方式“手动插入”:\begin{align}
\label{eq:insert_phase}
    \E_r(z,t) &= |\E_0|\cos\left(\betag(\omega)z-\omega t+\phi\right) \Rightarrow \\ \nonumber \E_r'(z,t) &= |\E_0|\cos\left(\betag(\omega)z-\omega t+\phi+\phi_0^{\textcolor{Red}{\swarrow}}\right).   
\end{align}
使用复电场,相同的操作可以通过复数乘法来完成,因为
\begin{align}
\label{eq:insert_phase_complex}
    \E(z,t) &=\E_0 \exp\left( i\betag(\omega)z-i\omega t+ i\phi\right)\Rightarrow \\ \nonumber \E'(z,t) &=  \E_0 \exp\left( i\betag(\omega)z-i\omega t +i\phi\right)\exp\left( i\phi_0\right) \\ \nonumber
    &=\E_0 \exp\left( i\betag(\omega)z-i\omega t +i\phi+ i\phi_0\right) \\ \nonumber
    \E_r'(z,t) &= \real\left\{  \E'(z,t)   \right\}. 
\end{align}

此外,由于振荡的实电场的瞬时功率与其平方成正比,平均功率可以通过积分计算或通过复电场绝对值平方的一半来计算,
\begin{align}
\label{eq:average_power}
    \langle \E_r^2 \rangle_T&= \frac{1}{T}\int_{0}^{T} \E_r^2 dt = \langle\real\left\{\E\right\}^2\rangle_T =\frac{1}{4}\langle\left(\E+\E^*\right)^2\rangle_T\\ \nonumber
&=\frac{1}{4}\langle \E^2+\E^{*2}+2|\E|^2 \rangle_T=\frac{1}{2} \langle|\E|^2\rangle_T=\frac{1}{2}|\E|^2.
\end{align}
使用复电场替代“手动插入”和积分的便利性足以使得采用复电场进行计算并仅在必要时提取实电场变得值得。因此,本简介主要利用复电场,但强调这些仅是有用的数学抽象,而实电场具有物理意义,因为它们直接决定了电荷的加速。
\subsection{实际电场和电场包络}
无线电信号用于Wi-Fi的载波频率约为5 GHz,而最先进的示波器可以测量频率高达100 GHz的电场。相比之下,激光脉冲的电场通常在100 THz以上的载波频率下振荡。因此,以实际复电场 \(\E(z,t) = \E_0 \exp\left( i\betag(\omega_0)z - i\omega_0 t\right)\) 进行计算和表达结果往往不够方便,因为以 \(\omega_0/2\pi\) 次每秒发生的快速电场振荡实际上是无法检测到的。相反,我们可以将复电场的包络定义为
\begin{align}
\label{eq:envelope}
    \A(z,t) &=a\cdot\E(z,t)\cdot e^{-i(\betag(\omega_0)z-\omega_0t)},
\end{align}
并用它进行计算。在这里,\( a = \sqrt{0.5 \epsilon_0 n c A_{eff}} \),其中 \(\epsilon_0\) 是真空的介电常数,\( n \) 是介质的折射率,\( c \) 是光速,\( A_{eff} \) 是光场横截面的有效面积,是一个归一化常数。将 \(\E\) 按 \( a \) 缩放确保 \(\A\) 具有 \(\sqrt{W}\) 的单位,因此 \( |\A|^2 \) 具有 \( W \) 的单位。通过“提取”快速且不可检测但又是\emph{可预测的}时间和空间振荡,确定电场由于线性和非线性效应而\emph{变化}的方式变得更加简单。有关 \(\E\) 和 \(\A\) 之间差异的示例,请参见此 \href{https://www.desmos.com/calculator/rsw2fn5af6}{交互式图表}。


\section{傅立叶变换}
%Define the Fourier Transform so going from time to frequency and back is well-behaved. 

在本教程中,傅里叶变换及其逆变换被定义为:
\begin{align}
    \Tilde{\E}(z,\omega) &= \FT\left\{\E(z,t)\right\} = \int_{-\infty}^{\infty} \E(z,t) e^{i\omega t} dt, \\ \nonumber
    \E(z,t) &= \IFT\left\{\Tilde{\E}(z,\omega)\right\} = \frac{1}{2\pi} \int_{-\infty}^{\infty} \Tilde{\E}(z,\omega) e^{-i\omega t} d\omega.
\end{align}
使用 $\exp(i\omega t)$ 的傅里叶变换约定,而不是 $\exp(-i\omega t)$,是因为沿着正 z 方向传播的复平面波由 $\exp(i\betag(\omega_0)z-i\omega_0 t)$ 描述。因此,计算,

\begin{align}
    \FT\left\{\exp(i\betag(\omega_0)z-i\omega_0 t)\right\} &= \int_{-\infty}^{\infty} e^{i\betag(\omega_0)z-i\omega_0 t} e^{i\omega t} dt, \\ \nonumber
      &= \int_{-\infty}^{\infty} e^{i\betag(\omega_0)z-i(\omega_0-\omega) t} dt \\ \nonumber
      &= e^{i\betag(\omega_0)z}\delta(\omega-\omega_0),
\end{align} 
显示复平面波的傅里叶变换产生一个以正载波角频率 $\omega$ 为中心的 δ 函数。如果在傅里叶变换中使用 $\exp(-i\omega t)$ 并将其应用于沿正 z 方向传播的复平面波,结果将包含 $\delta(\omega_0+\omega)$,这意味着 δ 函数是以负载波角频率为中心的。后一种方法使得在涉及沿 z 方向传播的复平面波的傅里叶变换的计算中更复杂,因此采用前一种约定。

\section{脉冲}
在介质中具有有限持续时间的电磁脉冲可以视为一组不同的复平面波的无限和,如下所示:
\begin{align}
    \label{eq:pulse}
    \E(z,t) &= \frac{1}{2\pi}\int_{-\infty}^{\infty} |\Tilde{\E}(z,\omega)| e^{i\betag(z,\omega)z-i\omega t+i\phi(z,\omega)} d\omega \\ \nonumber
    \E(z,t) &= \frac{1}{2\pi}\int_{-\infty}^{\infty} \Tilde{\E}(z,\omega) e^{i\betag(z,\omega)z-i\omega t} d\omega.
\end{align}
方程~\ref{eq:pulse} 提供的关键见解是,光信号的强度、形状和颜色的任何变化都必须源于改变一组时间频率分量的幅度 $|\Tilde{\E}(z,\omega)|$、相位 $\phi(z,\omega)$ 或空间频率 $\betag(z,\omega)$。即使是非线性效应,这些效应可以以令人惊讶的方式改变激光脉冲,本质上也只不过是改变这三个参数。


\section{啁啾和延迟}
%Highlight that how the phase changes with time determines instantaneous frequency, which is super important for understanding pulse evolution. Explain that the change in phase w.r.t. frequency determines the time shift for each frequency.
考虑在 $z=0$ 处给出的复电场

\begin{align}
\label{eq:chirp_example}
    a\E(t) &= \A(t)\exp\left(  -i\left(\omega_0 +\frac{C}{2}T \right)T   \right).
\end{align}

如果 $C=0$,电场的相位线性变化,这意味着它以固定的载波频率 $\omega_0$ 振荡。如果 $C>0$,方程~\ref{eq:chirp_example} 表明载波频率会随时间增加,而 $C<0$ 则意味着载波频率会降低。请参见 \href{https://www.desmos.com/calculator/gd7s8nhfdn}{这个互动图} 以获取说明。电场的“瞬时角频率”定义为

\begin{align}
\label{eq:chirp_definition}
    \delta\omega(T) &= -\partial_T\phi(T),
\end{align}

其中 $\phi(T)$ 是电场的相位随时间的函数。负号在时间导数前面是为了确保沿 z 方向传播的复平面波的瞬时角频率被正确计算为 $+\omega_0$,其表达式为 $\exp(i\beta(\omega_0)z-i\omega_0 t)$。

瞬时频率随时间变化的电场被称为“啁啾”。在频率低于(高于)其载波频率的时间段称为“红啁啾”(“蓝啁啾”)。从“红色”变为“蓝色”的啁啾称为“增加”,而从“蓝色”变为“红色”的啁啾称为“减少”。

理解已知电场的不同持续时间可以具有不同的瞬时频率,并且这些频率可以通过计算相位的负导数从方程~\ref{eq:chirp_definition} 中获得,对于理解许多线性和非线性效应至关重要。例如,如果某一介质中的光速使得高频(即“更蓝”的频率)比低频(即“更红”的频率)传播得更快,那么通过这种材料传播的光脉冲将在前面形成蓝啁啾,而在后面形成红啁啾。

正如对时间相位的导数可以提供有关瞬时频率的信息一样,可以对光谱相位关于频率的导数进行计算,以确定特定频率的时间延迟。考虑在 $z=0$ 处给出的复电场包络的光谱

\begin{align}
\label{eq:spectrum_time_example}
    \Tilde{\A}(\omega) &= \Tilde{\A}_0(\omega)\exp\left( i\left(\frac{B}{2}(\omega-\omega_0)^2 \right)   \right).
\end{align}

假设 $B>0$,方程~\ref{eq:spectrum_time_example} 表明围绕载波角频率 $\omega_0$ 的频率分量的相位随着与载波的距离增加而以二次形式增加。或者,可以将方程~\ref{eq:spectrum_time_example} 解释为:对于 $\omega_0$ 以下的角频率,相位 \emph{减少},而对于 $\omega_0$ 以上的角频率,相位 \emph{增加}。受方程~\ref{eq:chirp_definition} 启发,我们可以计算

\begin{align}
\label{eq:delay_definition}
    \delta t(\omega) &= \partial_\omega\phi(\omega),
\end{align}

对于方程~\ref{eq:spectrum_time_example},得出

\begin{align}
    \delta t(\omega) &=  B(\omega-\omega_0),
\end{align}

这表明 $\omega_0$ 以下的角频率经历负时间延迟(使其更早到达),而 $\omega_0$ 以上的角频率经历正时间延迟(相当于延迟)。请注意,在方程~\ref{eq:delay_definition} 中没有需要改变符号的要求以保持一致。

与方程~\ref{eq:chirp_definition} 相比,方程~\ref{eq:delay_definition} 对于计算的实用性较差,但有关相位随角频率的降低意味着提前到达时间,而相位随角频率的增加意味着延迟到达的见解在分析第~\ref{ch:Dispersion} 章中色散的影响时是有帮助的。有关改变光谱相位与时间延迟之间关系的说明,请参见 \href{https://youtu.be/E3S0BQiy3p8}{此视频教程}。






    \chapter{损耗和增益}
\label{ch:attenuation}


光在玻璃介质中传播时会自然衰减,因为杂质或晶格缺陷可能会将部分电磁场散射离开原来的传播方向。由于如果光的强度较低,非线性光学效应的影响会减小,因此理解衰减的影响对于理解方程~\ref{eq:GNLSE} 至关重要。


\section{损耗的物理起源}
当光线在波导(如光纤)中传播时,玻璃中的化学杂质、晶格中的晶体缺陷或波导中的紧密弯曲会导致衰减。此外,如果光的载波频率与晶格中化学键的振动频率相匹配,光就可以直接转化为热量。有关光如何在理想介质中传播的直观图片,请参阅 \href{https://www.youtube.com/watch?v=QCX62YJCmGk&list=PLZHQObOWTQDMKqfyUvG2kTlYt-QQ2x-ui}{本视频系列}。


\section{损耗的描述}

考虑公式 ~\ref{eq:GNLSE},其中除了 $\alpha$ 之外的所有参数都等于零,在这种情况下

\begin{align}
    \label{eq:attenuation}
    \partial_z\A &= \frac{\alpha}{2} \A \\ \nonumber
    \A(z,T)&=\A(0,T)\exp\left( \frac{\alpha}{2}z \right). 
\end{align}

当 $\alpha<0$ 时,公式~\ref{eq:attenuation}意味着场会随距离衰减,而 $\alpha>0$ 则意味着场会增益。请注意,场的功率与场的绝对平方成正比,因此

\begin{align}
    \label{eq:attenuation_power}
    P(z,t)=|\A(z,t)|^2&=|\A(0,t)|^2 \exp\left( \alpha z \right). 
\end{align}
在公式 ~\ref{eq:GNLSE}、公式 ~\ref{eq:attenuation}和公式 ~\ref{eq:attenuation_power}中,$\alpha$是 “功率损耗系数”,它决定了场的\emph{总功率}在传播一定距离后发生了多少因数的变化。其他作者将 $\alpha$ 定义为“场损耗系数”,其决定了\emph{场}在传播一定距离后发生的 $e$ 的$-\alpha L$次方的变化量。在这种情况下,方程~\ref{eq:attenuation_power} 中的指数部分将包含 $2\alpha z$,而不是 $\alpha z$。

\section{dB的单位换算}
\begin{table}[]
    \centering
    \begin{tabular}{ c|c|c|c|c|c|c|c|c|c|c|c|c|c|c|c|c|c }
\label{tab:dB}
 缩放因子 &0.01&0.05 & 0.1 &0.125 &0.25&0.4 &0.5 & 0.8&1&1.25 &2 &2.5 &4 &8 &10 &20 & \\  \hline
 dB change & -20 &-13 &-10 & -9&  -6&-4& -3&-1&0&1 &3 &4 &6 &9 & 10&13 
\end{tabular}
    \caption{缩放因子和相应变化(以 dB 为单位)的示例。}
    \label{tab:dB}
\end{table}

在实际应用中,通常以 “分贝”(dB)为单位报告衰减或增益。例如,如果一个信号最初的功率为 60~mW,而通过介质传播后的功率仅为 130~$\mu$W,则其功率变化了

\begin{align}
    \Delta P [dB] &= 10 \log_{10}\left(\frac{P_{final}}{P_{initial}} \right)= 10 \log_{10}\left(\frac{0.130 mW}{60 mW} \right) = -26.64 dB.
\end{align}
反之,如果功率再提升 15 分贝,则新功率为
\begin{align}
\label{eq:boost}
    P_{final}&=P_{initial}\cdot10^{\frac{\Delta dB}{10}}=0.13mW\cdot10^{\frac{15}{10}} = 4.11 mW.
\end{align}


注意,$10\log_{10}(10)=10$~dB,$10\log_{10}(0.1)=-10$~dB,$10\log_{10}(2)\approx3$~dB,且 $10\log_{10}(0.5)\approx-3$~dB。因此,如果功率放大了 $20=10\cdot 2$ 倍,这相当于增大了 $10\log_{10}(10\cdot 2) = 10\log_{10}(10)+10\log_{10}(2) = 13$~dB。更多示例参见表~\ref{tab:dB}。

以下是上述内容的翻译:

衰减系数 $\alpha$ 通常以 dB/km 为单位表示。例如,通信用单模光纤在 193~THz 附近(对应波长约为 1550~nm)的典型衰减系数为 $-0.22$ dB/km。这意味着一根 100 公里的光纤将使光功率变化 $-22$ dB,对应于功率约减少 158.5 倍。如果 $\alpha=-0.22$ dB/km,则在方程~\ref{eq:GNLSE} 中使用的 $\alpha$ 值(以 “每公里的 e 次方衰减” 为单位,假设 $z$ 以公里为单位)可以通过以下方法计算:
\begin{align}
    \exp\left(\alpha_{\text{Factors of $e$ per km}}z\right) &= 10^{ \frac{\alpha_{dB/km}}{10}z} \\ \nonumber
    \alpha_{\text{Factors of $e$ per km}}z &= \ln\left(10^{ \frac{\alpha_{dB/km}}{10}z} \right)\\ \nonumber
    &= \frac{\alpha_{dB/km}}{10}z \ln(10)\\ \nonumber
    \alpha_{\text{Factors of $e$ per km}} &= 0.23\cdot \alpha_{dB/km}\\ \nonumber
    \alpha_{\text{Factors of $e$ per km}}\cdot4.343&=\alpha_{dB/km}.
\end{align}


\section{在dBm下测量功率}

光信号的功率通常以“dBm”为单位表示,这代表相对于“毫瓦(mW)”的分贝功率。例如,40~mW 对应于

\begin{align}
    P [dBm] &= 10\cdot\log_{10}\left(\frac{40mW}{1mW} \right)=16dBm,
\end{align}
反之亦然
\begin{align}
    \label{eq:dBm_rev}
    P [mW] &= 10^{\frac{16dBm}{10}}mW=40mW.
\end{align}

使用“dBm”很方便,因为尽管信号较弱,几纳瓦的信号仍可被探测到,而经过\href{https://youtu.be/Eh5CHRWFT-M}{啁啾脉冲放大}等处理后的脉冲峰值功率可以达到兆瓦甚至吉瓦量级~\cite{Chirp_STRICKLAND_Nobel_prize}。它还使得损耗和增益的影响计算比使用方程~\ref{eq:boost} 更为简单,因为一个初始功率为 $0.13$~mW=$-8.86$~dBm 的信号,若增益为 15~dB,则最终功率为 $-8.86~dBm+15~dB=6.13$~dBm。


\section{常见的 dB 和 dBm 错误}

\begin{enumerate}
    \item \textbf{将 dBm 相加}:Alice 想确定两个激光器的总功率,以 dBm 为单位。她测得第一个激光器的功率为 3 dBm,第二个激光器的功率为 6 dBm。
    \begin{itemize}
    \item[\redcross] Alice 计算 $P_{tot} [dBm] = 3 dBm + 6 dBm = 9 dBm$,这是错误的,因为 3~dBm = 2~mW 且 6~dBm = 4~mW,但 9~dBm = 8~mW。直接相加 dBm 值对应于将线性单位下的功率相乘:$2~mW \cdot 4~mW = 8~mW^2$!
    \item[\greencheck] Alice 首先将两个测量值转换为线性单位(3~dBm=2~mW,6~dBm = 4~mW)。然后,她将线性值相加得到总功率 6~mW,最后将结果转换为 dBm:$10 \cdot \log_{10}(6mW/1mW) = 7.78~dBm$。
    \end{itemize}

    \item \textbf{混淆 dB 和 dBm}:Bob 想知道某个激光器的功率(以 mW 为单位)。他的同事报告测得功率为“13~dB”。
    \begin{itemize}
    \item[\redcross] Bob 将 $13~dB$ 代入方程~\ref{eq:dBm_rev} 而不是 $16~dBm$,并得到结果 $20~mW$。这是错误的,因为“13~dB”是无量纲比值,而不是功率的度量!
    \item[\greencheck] Bob 询问同事是否实际上是指“13~dBm”。同事可能只是口误,但也有可能是由于功率计的显示模式设置错误,因此报告了相对于某个预设值的当前功率,导致显示“13~dB”。
    \end{itemize}

   \item \textbf{另一种混淆 dB 和 dBm}:Charlie 测得某激光器的功率为 10~dBm。他的同事要求他将功率降低“3~dBm”。
    \begin{itemize}
    \item[\redcross] Charlie 将输出功率从 10~dBm 调至 7~dBm。这是错误的,因为他减少了 3~dB,而“3~dBm”意味着减少 2~mW。
    \item[\greencheck] Charlie 询问同事是否是指减少 3~dB,或确实希望减少 2~mW。前者更有可能,因为后者(严格来说并非错误)是一种不常用且可能引起混淆的术语使用方式。
    \end{itemize}
    
    \item \textbf{不正确的平均值计算}:David 想确定三个激光器的平均功率,以 dBm 为单位,分别为 -1~dBm、4~dBm 和 6~dBm。
    \begin{itemize}
    \item[\redcross] David 计算 $P_{avg}=(-1~dBm+4~dBm+6~dBm)/3=4~dBm$。这样做并没有得到“常规”平均值,而是三个值的\href{https://en.wikipedia.org/wiki/Geometric_mean}{几何平均值}:$(-1~dBm+4~dBm+6~dBm)/3 = (0.79~mW \cdot 2.51~mW \cdot 7.94~mW)^{1/3}=2.5~mW = 4~dBm$。在某些情况下此值有一定参考意义,但这并不是 David 想要的“常规平均值”!例如,当一个大团队希望表征其光学产品的性能以便销售时,验证功率平均值的计算是否一致非常重要!
    \item[\greencheck] David 将三个功率转换为线性单位,计算平均值为 $(0.79~mW+2.51~mW+7.94~mW)/3=3.75~mW=5.74~dBm$。
    \end{itemize}
    
\end{enumerate}





    \chapter{Dispersion}
\label{ch:Dispersion}
Chapter will explain how the propagation of pulses is affected by dispersion to different orders.



\section{$\beta_0$}
Explain how it alters the spatial phase
\section{$\beta_1$}
Explain how it alters pulse propagation. 
\section{$\beta_2$}
Explain how it causes pulse broadening. Heat equation etc.
\section{$\beta_n$}
Generalize insight.


\subsection{Theoretical FBG model}

\begin{align}
    \nabla^{2}\tilde{\boldsymbol{E}}+\frac{\omega^{2}}{c^{2}}\tilde{\boldsymbol{E}}+\text{\ensuremath{\mu_{0}}}\omega^{2}\tilde{\boldsymbol{P}}&=0,
\end{align}
    \chapter{Self Phase Modulation}
\label{ch:SPM}

Explain that SPM is the "most basic" nonlinear effect. 

\section{Phase change across pulse}

\section{Spectral broadening}

\section{Different pulse shapes}


\begin{align}
    \nabla^{2}\tilde{\boldsymbol{E}}+\frac{\omega^{2}}{c^{2}}\tilde{\boldsymbol{E}}+\text{\ensuremath{\mu_{0}}}\omega^{2}\tilde{\boldsymbol{P}}&=0,
\end{align}
    \chapter{Four Wave Mixing}
\label{ch:FWM}

Explain that FWM occurs when multiple frequencies modulate the refractive index. 




\section{Electrical modulation}

\section{Bessel function approach}





\begin{align}
    \nabla^{2}\tilde{\boldsymbol{E}}+\frac{\omega^{2}}{c^{2}}\tilde{\boldsymbol{E}}+\text{\ensuremath{\mu_{0}}}\omega^{2}\tilde{\boldsymbol{P}}&=0,
\end{align}
    \chapter{The Raman effect}
\label{ch:Raman}

This chapter explains how to understand the Raman-effect. Link to Desmos notebooks. 

\section{Microscopic picture}

\section{Response function and convolution}

\section{Impact}


\begin{align}
    \nabla^{2}\tilde{\boldsymbol{E}}+\frac{\omega^{2}}{c^{2}}\tilde{\boldsymbol{E}}+\text{\ensuremath{\mu_{0}}}\omega^{2}\tilde{\boldsymbol{P}}&=0,
\end{align}

    \chapter{Exotic pulses}
\label{ch:Exotic}

For a fiber, where $\gamma\neq 0$ and $\betag_2\neq 0$, a pulse launched into it can evolve in a number of surprising ways. Essentially, the nonlinearity will change the local frequency of the pulse based on its power profile, while dispersion will cause different frequencies to advance or delay relative to the carrier frequency, which in turn, alters the power profile. This chapter explores the interplay between these two effects for different values of $\gamma$ and $\betag_2$. 


\section{The Fundamental Soliton}
\label{sec:soliton}
Consider a fiber for which $\gamma>0$. As illustrated in Fig.~\ref{fig:chirp_profiles}, this will cause leading(trailing) pulse edges to develop a red(blue) shift. As explained WHERE?!?!?! $\betag_2<0$ implies that blue light propagates faster than red light, thus causing the leading(trailing) edge of a pulse will become more blue(red). If $\gamma>0$ makes the front(back) of the pulse more red(blue) based on the instantaneous power profile, while $\betag_2<0$ makes the front(back) more blue(red) according to the 2nd time derivative of the field, it's natural to ask if there exists a pulse envelope, where the two effects balance exactly at every instant, so that the shape of the pulse is unaltered as it propagates forward. For such a pulse, the field envelope should be independent of distance, while the phase should be independent of time, such that
\begin{align}
    \A(z,T) &= V(T)e^{i\phi(z)}
\end{align}
solves 
\begin{align}
\label{eq:NLSE_soliton}
    \partial_z\A &=  -i  \frac{\betag_2}{2}\partial_T^2\A+i\gamma|\A|^2\A.
\end{align}
As explained in \href{https://github.com/OleKrarup123/NLSE-vector-solver/blob/main/TutorialVideos/Soliton-Video/Fundamental_soliton_derivation.pdf}{this derivation}, it can be shown that the solution is
\begin{align}
    \A(z,T) &= \sqrt{\frac{|\betag_2|}{\gamma T_0^2}}\cdot\text{sech}\left(\frac{T}{T_0}\right)\exp\left(i\frac{|\betag_2|}{2T_0^2}z\right),
\end{align}
where $T_0$ is the time at which the field of the pulse has decreased to 64.8\% of its peak value. This stable pulse characterized by a hyperbolic secant envelope is referred to as a "fundamental soliton". See Fig.~\ref{fig:gauss_sech} for a comparison of a Gaussian pulse to a hyperbolic-secant pulse. Note that the peak power of the hyperbolic-secant pulse, $\A_{max}$, must be chosen to exactly equal the characteristic amplitude, $\A_{char}=\sqrt{|\betag_2|/\gamma T_0^2}$, for stable propagation to occur!
\begin{figure}
    \centering
    \includegraphics[width=1\linewidth]{figures/gauss_sech_comparison.png}
    \caption{Comparison of a Gaussian pulse to a hyperbolic-secant pulse on a linear scale in a) and a logarithmic scale in b), which emphasizes the comparatively more intense tails of the hyperbolic-secant.  }
    \label{fig:gauss_sech}
\end{figure}
\subsection{Solitons in telecommunications}
As $\gamma>0$ and $\betag_2<0$ in silica fibers for near-infrared frequencies close to 193~THz ($\approx$1550~nm), fundamental solitons were once of great interest in telecommunications because their resistance to dispersion and constant shape reduced inter-symbol interference. However, the same improvements in electronic dispersion compensation mentioned in Subsection~\ref{subsec:ZDF} have rendered the use of solitons for optical fiber communication obsolete. Instead, pulses with a so-called "\href{https://www.youtube.com/watch?v=Qe8NQx4ibE8}{root-raised-cosine}" field envelope are used.   


\section{Higher order Solitons}
If the peak power of the hyperbolic secant pulse smaller than $P_{char}=|\A_{char}|^2=|\betag_2|/\gamma T_0^2$, the nonlinear effect will be too weak compared to dispersion for a fundamental soliton to form. The pulse wivll thus simply broaden in the time domain as it propagates forward. If the peak power is increased beyond $P_{char}$, the pulse will exhibit "oscillations" as it evolves. For the special cases of $\A_{max}$ being an integer multiple of $\A_{char}$, the oscillating soliton evolution will be particularly well-behaved. See Fig.~\ref{fig:Soliton_comparison}~a-b) for an example of the temporal and spectral evolutions of a soliton for which $\A_{max}=3\A_{char}$. See \href{https://youtu.be/KAZ7pCQ-x8Y}{this video tutorial} for more information on solitons in fibers. 

\subsection{Soliton fission}
\label{subsec:fission}
Fundamental- and higher order solitons can be seen as "fixed points" of Eq.~\ref{eq:NLSE_soliton}. However, Eq.~\ref{eq:NLSE_soliton} itself is only an approximation to Eq.~\ref{eq:GNLSE}, which provides a more general description of the physics affecting the propagation of an optical pulse in a dispersive and nonlinear medium. It turns out that the presence of effects other than $\gamma>0$ and $\betag_2<0$ will eventually disturb the evolution of an initially solitonic pulse. In Fig.\ref{fig:Soliton_comparison}~c-d), the presence of $\betag_3>0$ causes a higher order soliton to "fission" into two less powerful ones at new frequencies due to FWM after propagating a distance of 0.5 meters. Other effects, such as Self-Steepening, the Raman effect or even Modulation Instability caused by optical noise propagating along with an otherwise ideal soliton pulse can similarly give rise to soliton fission. Thus, much like a pencil balanced on its tip, solitonic propagation should be viewed as an unstable equilibrium. See \href{https://youtu.be/tHpIR2Kuxp0}{this video tutorial} for more details on soliton fission.
\begin{figure}
    \centering
    \includegraphics[width=1\linewidth]{figures/Soliton_comparison.png}
    \caption{a) Temporal evolution of an $N=3$ soliton for which $\A_{max}=3\A_{char}=3\sqrt{|\betag_2|/\gamma T_0^2}$. The length of the fiber has been chosen to equal the oscillation period of the soliton. b) Spectrum evolution of the $N=3$ soliton. c-d) Respectively, the temporal and spectral evolutions of the same $N=3$ soliton as in a-b), but where the fiber has $\betag_3>0$, causing the pulse to "fission", thereby illustrating the unstable nature of solitonic propagation. Figures generated using the numerical simulation in \href{https://colab.research.google.com/drive/123pT-IsLWIEZY9XW3-1WzkTXfg1IEkkD?usp=sharing}{this interactive notebook}, which the reader is encouraged to experiment with.}
    \label{fig:Soliton_comparison}
\end{figure}





\section{Optical Wave Breaking}
When an ocean wave approaches a beach, water begins to "pile up" on its leading edge, eventually causing the wave to "break" before crashing onto the shore. A similar phenomenon can be observed with optical waves in nonlinear fibers where $\gamma>0$ and $\betag_2>0$. Instead of the local color-changes from nonlinearity and dispersion balancing as in Sec.~\ref{sec:soliton} when $\betag_2$ was negative, they now "cooperate". The nonlinearity makes the front(back) of the pulse more red(blue) and dispersion causes red(blue) light to move faster(slower) than the carrier. The result is that any pulse will quickly broaden in the time domain. Often, this happens in such a way that power gradually "piles up" in both the front and back of the pulse, leading to steeper power slopes, which generate even larger chirps causing more rapid temporal broadening due to dispersion. When the slope steepness gets sufficiently large, dispersion will "launch" the newly generated frequencies away from the main pulse in a manner analogous to a crashing water wave. Figure~\ref{fig:OWB_and_similariton}~a-b) illustrates this effect called "Optical Wave Breaking" (OWB). While the changes to incident pulses induced by OWB can be detrimental, they can also be beneficial if one desires an optical pulse with steep slopes and approximately constant peak power. See \href{https://youtu.be/XEx6lOf6f40}{this video tutorial} for further details on OWB. 

\subsection{Similaritons}
When $\gamma>0$ and $\betag_2>0$, OWB broadens the pulse in the time domain, thereby reducing its peak power. If, additionally, $\alpha>0$, implying that the pulse is amplified as it moves forward, this loss of peak power is continuously replaced. It turns out that under these circumstances, any input pulse \emph{regardless of its initial shape} will evolve towards a parabolic power envelope as illustrated in Fig.~\ref{fig:OWB_and_similariton}~c) and a linearly changing chirp that goes from red in the front to blue in the back\cite{Similariton_evolution}. Both the peak power and duration of such a "similarition" will continuously increase with distance as explained in \href{https://youtu.be/ZtWIRaj5VV4}{this video tutorial}. Similaritons can arise in certain optical amplifiers and their linearly changing chirps make them ideal candidates for generating pulses with extremely high peak powers through the Nobel-Prize winning method of chirped pulse compression explained in \href{https://youtu.be/Eh5CHRWFT-M}{this video tutorial}.   

\begin{figure}
    \centering
    \includegraphics[width=0.57\linewidth]{OWB_and_similariton.png}
    \caption{a) Evolution of a triangular pulse through a medium where $\betag_2>0$. b) Evolution of the same pulse as in a) through a medium where $\betag_2>0$ and $\gamma>0$ causes power to "pile up" in the front and back of the pulse. c) Same conditions as in b), but where $\alpha>0$ causes the triangular pulse to evolve towards a parabola. Figures generated using the numerical simulation in \href{https://colab.research.google.com/drive/1qtMcXElXn4VBntfCgXIGGkyDfiGicElx?usp=sharing}{this interactive notebook}, which the reader is encouraged to experiment with.  }
    \label{fig:OWB_and_similariton}
\end{figure}


\section{Novel Solitons}
\subsection{Dark Solitons} 
Typically, optical pulses consist of a sudden increase in laser power preceded and followed by long durations of zero power. To the naked eye, such a pulse would be a bright flash, like a lamp being briefly switched on in a dark room. Such pulses can propagate stably through a nonlinear medium if the conditions described in Sec.~\ref{sec:soliton} are satisfied. Consider instead what might be called an "anti-pulse" in a nonlinear medium; a high power CW signal, which experiences a brief dip in its power, analogous to a bright lamp that is briefly switched off before being reactivated. It will consist of a decreasing leading edge and an increasing trailing edge. If $\gamma>0$ and $\betag_2>0$, the leading(trailing) edge becomes more blue(red), causing the light there to slow down(speed up). Similarly to Sec.~\ref{sec:soliton}, one can calculate that the field envelope for which nonlinearity and dispersion cancel exactly is a hyperbolic tangent, thereby leading to a stably propagating "Dark Soliton" described by 
\begin{align}
    \A(z,T) = \A_{char}\tanh\left(\frac{T}{T_0}\right)\exp\left(i\frac{|\betag_2|}{T_0^2}z \right).
\end{align}
See \href{https://youtu.be/MrNfI1_eTZ0}{this video tutorial} for more information on Dark Solitons.


\subsection{Raman Solitons}
If the Raman effect, which shifts the spectrum of a pulse towards lower frequencies, is present in a nonlinear medium where $\betag_2<0$, a "Raman Soliton" can form. This soliton retains the shape of its envelope, but undergoes a constant red-shift and obtains an increasingly large time delay since $\betag_2<0$ implies that red light moves more slowly than blue light as illustrated in Fig.~\ref{fig:dark_and_raman}c-d). Raman solitons often arise after soliton fission of pulses with durations on the scale of tens of femtoseconds. See \href{https://www.youtube.com/watch?v=K33YUfegL1w}{this video tutorial} for an explanation of Raman solitons.

\begin{figure}
    \centering
    \includegraphics[width=1\linewidth]{figures/dark_and_raman_soliton_combined.png}
    \caption{a) Fundamental dark soliton propagating stably through a medium where $\betag_2>0$ and $\gamma>0$. b) An N=3 dark soliton propagating through the same medium as in a) Instead of stable propagation, the central dip gives rise to additional ones. c) An N=3 soliton propagating through a medium where $\betag_2<0$ and $\gamma>0$. d) The same pulse as in c) propagating through the same medium except the Raman effect described by Eq.~\ref{eq:raman_basic} is taken into account, causing soliton fission and a self-redshifting Raman soliton to arise.  }
    \label{fig:dark_and_raman}
\end{figure}

\subsection{Vector Solitons}
In Sec.~\ref{Sec:XPM}, it was shown that two distinct frequencies of light can affect each others phases through the nonlinearity. Similarly, two different polarizations of light propagating in a nonlinear medium can affect each other's phases. Using a vectorial version of Eq.\ref{eq:GNLSE} where $\gamma>0$, $\betag_2<0$, and where the refractive index is different for light polarized along the x- and y-axes of the medium, one can obtain analytical expressions for so-called "Vector Solitons". The special cases for which this is possible turn out to be circularly polarized light and light polarized linearly at $45^{o}$ to the x-axis. See AGRAWAL!?!?!?! for more details. 







    \chapter{超连续谱生成}
\label{ch:supercontinuum}

前面的章节解释了单个线性和非线性效应及其相互作用,并总结了它们的影响,如表~\ref{tab:NL_summary}所示。对于具有载频使得$\betag_2\lesssim 0$且$\gamma>0$的高功率脉冲,以及持续时间低于约100~fs的情况,这些效应可能同时存在。在这种情况下,脉冲的演化可能非常复杂,并将其谱宽度扩展到其初始带宽的十到二十倍~\cite{supercontinuum_original_paper}。本章通过一个超连续谱生成的例子,阐述了如何通过之前讨论的效应理解这一过程。

\begin{table}[]
\begin{adjustwidth}{-2cm}{}
\begin{tabular}{ccccc}
\hline
\textbf{效应}       & \textbf{时域表现}                                                                                      & \textbf{光谱表现}                                                                                           & \textbf{对以下情况显著}                                                                           & \textbf{相关性}                                                                                                       \\ \hline
$\alpha>0$            & 增加功率.                                                                                           & 增加功率.                                                                                             & 放大器                                                                                         & \begin{tabular}[c]{@{}c@{}}非线性效应高度\\ 依赖于功率.\end{tabular}                                            \\ \hline
$\beta_2<0$           & \begin{tabular}[c]{@{}c@{}}脉冲前端(蓝)和后端(红)\\ 各自展宽.\end{tabular}            & \begin{tabular}[c]{@{}c@{}}光谱相位\\ 随与载频的\\ 距离变化.\end{tabular} & \begin{tabular}[c]{@{}c@{}}短的NIR脉冲\\ 在二氧化硅中.\end{tabular}                             & \begin{tabular}[c]{@{}c@{}}非线性效应在\\ $\beta_2+\gamma P\approx 0$时显著.\end{tabular}                      \\ \hline
$\beta_3$             & \begin{tabular}[c]{@{}c@{}}根据符号推迟或提前\\ 非载频成分.\end{tabular}    & \begin{tabular}[c]{@{}c@{}}光谱相位与\\ 载频的\\ 距离三次方变化.\end{tabular}        & \begin{tabular}[c]{@{}c@{}}载频接近\\ 零色散频率(ZDF).\end{tabular}                            & \begin{tabular}[c]{@{}c@{}}不同频率在时域中\\ 重叠导致FWM.\\ 孤子分裂.\end{tabular} \\ \hline
自相位调制(SPM) & \begin{tabular}[c]{@{}c@{}}脉冲前缘(红)和后缘(蓝)\\ 各自偏移.\end{tabular}                     & \begin{tabular}[c]{@{}c@{}}对称\\ 展宽.\end{tabular}                                            & 高功率脉冲                                                                                  & \begin{tabular}[c]{@{}c@{}}最基本的非线性效应. \\ 首先“启动”随着\\ 功率增大.\end{tabular}            \\ \hline
自陡化       & \begin{tabular}[c]{@{}c@{}}脉冲峰值延迟\\ 到后期时间,\\ 导致陡峭的后坡.\end{tabular} & \begin{tabular}[c]{@{}c@{}}偏向高频的\\ 展宽.\end{tabular}                 & \begin{tabular}[c]{@{}c@{}}短时脉冲与载频\\ 比较.\end{tabular} & \begin{tabular}[c]{@{}c@{}}SPM的细微修正。\\ 孤子分裂.\end{tabular}                            \\ \hline
拉曼效应     & 脉冲峰值红移。                                                                                  & \begin{tabular}[c]{@{}c@{}}偏向低频的\\ 展宽.\end{tabular}                  & \begin{tabular}[c]{@{}c@{}}极短脉冲在\\ 10-100fs量级.\end{tabular}     & \begin{tabular}[c]{@{}c@{}}拉曼红移效应\\ 可以超过SPM\\ 展宽.  \\ 孤子分裂.\end{tabular}            \\ \hline
\end{tabular}
\caption{不同线性和非线性效应的影响总结。}
\label{tab:NL_summary}
\end{adjustwidth}
\end{table}

\section{案例研究}
为了模拟超连续谱的生成,使用了表~\ref{tab:SC_params}中的参数。生成的超连续谱如图~\ref{fig:SC_combined}所示。图~\ref{fig:SC_combined} c)中的光谱表明,脉冲在前1~m的传播过程中经历了SPM效应,而图~\ref{fig:SC_combined} b)表明,孤子分裂发生在随后的区域。高功率脉冲具有比初始脉冲小得多的持续时间,它沿着抛物线向后延伸,形成一个拉曼孤子,并不断发生红移。线性"走离"的低功率光可能是由FWM生成的,这一过程发生在分裂的孤子与初始脉冲残余部分重叠时。有关此超连续谱的详细分析,请参阅\href{https://youtu.be/-GDsMDpC3oA}{此视频教程}。

\begin{table}[]
\centering
\begin{tabular}{cc}
\textbf{参数}                      & \textbf{数值}                                  \\ \hline
时间点数                           & $2^{14}$                                          \\ 
时间分辨率 {[}fs{]}                & 1.8                                             \\ \hline
脉冲类型                              & Sech                                            \\ 
持续时间 {[}fs{]}                       & 166.79                                          \\ 
峰值功率 {[}W{]}                      & 50                                              \\ 
载频 {[}THz{]}                 & 282.823 (=1060~nm)                         \\ \hline
$\alpha$ {[}dB/km{]}                    & 0                                               \\ 
$\betag_2$ {[}s\textasciicircum{}2/m{]} & -3.051721e-27                                   \\ 
$\betag_3$ {[}s\textasciicircum{}3/m{]} & 7.29029e-41                                     \\ 
$\betag_4$ {[}s\textasciicircum{}4/m{]} & -1.08817e-55                                    \\ 
$\betag_5$ {[}s\textasciicircum{}5/m{]} & 2.8941e-70                                      \\ 
$\betag_6$ {[}s\textasciicircum{}6/m{]} & 4.8348e-89                                      \\ 
$\betag_7$ {[}s\textasciicircum{}7/m{]} & -1.1464e-113                                    \\ 
$\betag_8$ {[}s\textasciicircum{}8/m{]} & 1.8802e-128                                     \\ 
$\betag_9$ {[}s\textasciicircum{}9/m{]} & -1.5054e-143                                    \\ 
$\gamma$ {[}1/W/m{]}                    & 0.09                                            \\ 
自陡化                         & 开启                                              \\ 
拉曼模型                             & 方程~\ref{eq:raman_basic}
\end{tabular}
\caption{用于生成超连续谱的模拟参数,取自~\cite{supercontinuum_paper}}
\label{tab:SC_params}
\end{table}

\begin{figure}
    \centering
    \includegraphics[width=0.9\linewidth]{figures/SC_combined.png}
    \caption{a) 使用表~\ref{tab:SC_params}中列出的参数比较特征长度。具有短特征长度的效应将首先变得显著。b) 脉冲的时间演化,显示在约1~m的距离处发生孤子分裂,之后是FWM并生成拉曼孤子,逐渐向后延伸。c) 脉冲的光谱演化。d) 初始时刻$z=0$的光谱图。e) $z=5$~m时的光谱图。}
    \label{fig:SC_combined}
\end{figure}

\section{你的实验}
为了进一步探索图~\ref{fig:SC_combined}中模拟的超连续谱特性,请打开生成该图的\href{https://colab.research.google.com/drive/1HvA8F8yzEq-9fahuI4z2KhT-YhdRAXgt?usp=sharing}{Colab笔记本},并进行以下实验,解释脉冲及其光谱演化的不同。每次实验前,请写下你对模拟结果的预期,以便与实际结果进行对比。请注意,在每次实验前“重置”参数到默认值:

\begin{enumerate}
\item \textbf{无非线性效应}。当前模拟使用了$\gamma>0$。将其改为$\gamma=0$。结果是否表明非线性效应对脉冲的时间演化有重要影响?

\item \textbf{无自相位调制效应}。当前模拟考虑了自相位调制的影响。关闭此效应后,观察其对结果的影响。

\item \textbf{负$\alpha$值}。当前模拟中使用$\alpha=0$。将其改为$\alpha=-1$~dB/m。观察该修改对结果的影响。

\item \textbf{正的$\betag_2$值}。当前模拟中使用$\betag_2<0$。将$\betag_2$的符号改为正值。

\item \textbf{仅$\betag_2<0$}。当前模拟中使用了$\betag_n\neq 0$($n>2$)。将$\betag_n = 0$($n>2$),运行模拟并解释脉冲及其光谱演化的变化。

\item \textbf{负$\betag_3$值}。当前模拟中使用$\betag_3>0$。将$\betag_3$改为正值。注意:为了确保正确绘制时间演化图,可能需要将时间偏移从-1.75~ps改为+1.75~ps。你能解释为什么当$\betag_3<0$时不会出现拉曼孤子吗?提示:使用公式~\ref{eq:ZDF}计算零色散频率并考虑符号变化对其的影响。

\item \textbf{修改拉曼模型}。当前模拟使用了方程~\ref{eq:raman_basic}来建模拉曼效应。根据笔记本中的提示,使用方程~\ref{eq:raman_new}、方程~\ref{eq:Raman_exact}或$f_R=0$来修改模型。
\end{enumerate}

    \chapter{Recommended material}
\label{ch:material}
The following is a list of useful and accessible resources on both optics, electronics and communication. 

\section{Optics}

\subsection*{Elements of Photonics}
This textbook by prof. Keigo Iizuka on the theory behind practical optical devices covers a wide range of topics. Highly recommended for its treatment of the gain and noise properties of the Erbium Doped Fiber Amplifier (EDFA), which is an essential instrument in modern optics and telecommunications. 


\subsection*{Modern Optics lectures by Prof. Partha Roy Chaudhuri}
This \href{https://www.youtube.com/watch?v=2WiMeh1Dxl8&list=PLbRMhDVUMngePMuAGeAUeGVuZffTFY-5i}{free series of 60 video lectures} covers the basics of Maxwell's Equations, polarization and birefringence as well as more advanced topics including wave guides and devices such as electro-optical modulators. Highly recommended for the well-structured and mathematically thorough treatment of all described subjects. 

\subsection*{Non-linear Optics lectures by Prof. Samudra Roy}
This \href{https://www.youtube.com/watch?v=EiIDScj124Q&list=PLbRMhDVUMngfBwyonVP8VIsabtnsV3GVv}{series of 60 video lectures} first explains essential concepts in linear optics followed by a detailed derivation of both 2nd-order and 3rd-order nonlinear effects starting from Maxwell's Equations. Second Harmonic Generation (SHG), Difference Frequency Generation (DFG), Optical Parametric Oscillators (OPO) along with other phenomena beyond the scope of this primer are covered. Particularly recommended for its explanation of the symmetry- and tensor properties of the $\chi^{(2)}$ and $\chi^{(3)}$ nonlinearities. 

\subsection*{Nonlinear Fiber Optics}
This textbook by Govind P. Agrawal explains the mathematics behind Eq.~\ref{eq:GNLSE} in great detail and with frequent reference to experimental results. Highly recommended for its in-depth treatment of advanced phenomena involving polarization, Stimulated Brillouin Scattering and novel fibers.

\subsection*{MIT "Demonstrations in Lasers and Optics" lecture series}
Presented by prof. Shaoul Ezekiel, \href{https://www.youtube.com/watch?v=1cEXNLP5uE0&list=PL4E7FAAD67B171EBC}{this series of 49 video lectures} contains practical demonstrations of fundamental optical effects, such as polarization, interference and diffraction in the context of free-space optics. Highly recommended for its systematic experiments.    

\subsection*{International School on Parametric Non Linear Optics lectures}
This \href{https://www.youtube.com/@ispnlo9041/videos}{series of 44 video lectures} delivered by experts in nonlinear optics. Highly recommended for covering advanced topics beyond what is usually described in ordinary course work.  

\subsection*{Les' Lab}
This \href{https://www.youtube.com/@LesLaboratory/videos}{YouTube channel} specializes in practical demonstrations of laser-technology, including dye lasers, spectrometers, electro-optical modulators, SHG and even super-continuum generation. Highly recommended for the illustrative demonstrations of relevant effects and the detailed explanations of the electronics required to operate optical devices.

\subsection*{YourFavouriteTA}
This \href{https://www.youtube.com/@yourfavouriteta/videos}{YouTube channel} by the author of this primer explains nonlinear optics through both practical experiments, theoretical derivations and numerical simulations. It aims to provide intuitive explanations of both linear and nonlinear processes usually described by intricate equations.  


\section{Electronics}
\subsection*{The Signal Path}
This \href{https://www.youtube.com/@Thesignalpath/videos}{YouTube channel} is dedicated to practical experiments, tear-downs and repairs of power supplies, signal generators, detectors, oscilloscopes, spectrum analyzers and other equipment commonly found in scientific and industrial laboratories. Particularly recommended for its videos on Radio-Frequency (RF) devices.

\subsection*{EMPossible}
This \href{https://www.youtube.com/@empossible1577/playlists}{YouTube channel} focuses on both the basics of electromagnetic theory and advanced topics such as RF waveguides, semi-conductor band structures and finite element analysis. Highly recommended for the broad range of concepts covered and the streamlined tutorials with excellent illustrations.  

\subsection*{Keysight Electronics Tutorials}
The \href{https://www.youtube.com/@KeysightLabs/videos}{YouTube Channel} of the electronics equipment company, Keysight, contains explanations focusing on measurements of electrical signals. Highly recommended for the focus on oscilloscopes as this is a commonly used tool for visualizing the power over time measured by photodiodes. 


\subsection*{Rohde\&Schwarz Electronics Tutorials}
This 
\href{https://www.youtube.com/watch?v=rUDMo7hwihs&list=PLKxVoO5jUTlvsVtDcqrVn0ybqBVlLj2z8}{YouTube playlist} produced by the electronics equipment company, Rohde\&Schwarz, provides in-depth tutorials on measurement and characterization of properties of electrical devices and signals, such as power, phase noise, amplifier noise factor and more. Many of these concepts are highly transferable to the field of optics. 



\section{Communication}


\subsection*{Ian Explains}
This \href{https://www.youtube.com/@iain_explains/videos}{YouTube channel} explains essential methods in digital and analog signal processing, communication and statistics. Highly recommended for its illustrative examples and straight-forward derivations.


\section{Simulation tools}

\subsection*{Octave Photonics}
This free, interactive, \href{https://www.octavephotonics.com/nlse}{browser based tool} for solving Eq.~\ref{eq:GNLSE} is great for visualizing the impact of essential phenomena like dispersion, soliton formation and the Raman effect. Highly recommended for its easy-to-use interface.

\subsection*{ssfm\_functions.py}
This \href{https://github.com/OleKrarup123/NLSE-vector-solver/tree/main}{open-source python library} for solving Eq.~\ref{eq:GNLSE} is created and maintained by the author of this primer. It provides a high degree of flexibility and customizability in setting up simulations using code. For example, arbitrary fiber links consisting of multiple spans with varying properties including different Raman models, input/output gain or dispersion compensation can be created using a for-loop, and simulation results are easily visualized using built-in plotting functions. The ability to modify the tool makes it ideal for graduate students in nonlinear optics, who often need to model novel and highly specific systems as part of their research.   

\subsection*{gnlse-python}
This \href{https://github.com/WUST-FOG/gnlse-python}{open-source python library} for solving Eq.~\ref{eq:GNLSE} has excellent \href{https://gnlse.readthedocs.io/en/latest/index.html}{documentation} and built-in visualization functions. An arXiv preprent explaining the implementation is also available~\cite{redman2021gnlsepythonopensourcesoftware}. 

\subsection*{GMMNLSE-Solver}
This \href{https://github.com/WiseLabAEP/GMMNLSE-Solver-FINAL}{open-source MATLAB library} allows one to solve a Multi-mode version of Eq.~\ref{eq:GNLSE} and model exotic phenomena such as multi-mode solitons. It is a highly advanced tool recommended for extreme use-cases if one is already very comfortable with the basic single-mode case presented in this primer. If configured correctly, it allows computations to be parallelized on the GPU, making it quite fast for such a complex simulation. 




    
  %\appendix
    %\input{ap-code}
  \backmatter
  \bibliographystyle{unsrturl}%plainurl
  \bibliography{refs.bib}
  
\end{document}
