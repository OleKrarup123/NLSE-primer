\chapter{Introduction}
\label{ch:Introduction}

Quick explanation of the purpose of this primer: Explain the basics of the NLSE with much more hand-holding than Agrawal but fewer advanced details. Remember to cite! Stick to basic 1D version to provide intuitive understanding of different effects that will help the reader to understand more advanced 2D effects.  

Explain that phenomena labelled SPM, XPM etc. should be thought of as a "taxonomy"; i.e. they are not "really" distinct. Rather the Kerr effect causing a power-dependent phase shift is the fundamental effect and the sub-effects are just simplified special cases.  

\section{FBG based sensors}

\subsection{Theoretical FBG model}

\begin{align}
    \nabla^{2}\tilde{\boldsymbol{E}}+\frac{\omega^{2}}{c^{2}}\tilde{\boldsymbol{E}}+\text{\ensuremath{\mu_{0}}}\omega^{2}\tilde{\boldsymbol{P}}&=0,
\end{align}